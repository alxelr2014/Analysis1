\chapter{Multivariable Calculus}
\thispagestyle{headings}
\section{Linear Algebra}

\begin{definition}[Normed vector space]
    Let \(V\) be a vector space. A \textbf{norm} is a real valued function \(\norm: V \to \Reals\) which has the following properties
    \begin{enumerate}
        \item \(\forall x \in V, \; \norm[x] > 0\).
        \item \(\norm[x] = 0 \implies x = 0\).
        \item \(\forall x \in V \; \forall \alpha \in \Field, \; \norm[\alpha x] = \abs[\alpha] \norm[x]\).
        \item \(\forall x,y \in V \; \norm[x+y] \leq \norm[x] + \norm[y]\).
    \end{enumerate}
\end{definition}

Each normed vector space induces a metric space \(\metricSpace{V}{d}\) where \(\func{d}{x,y} = \norm[y - x]\).

\begin{definition}
    Assume \(V\) is a vector space and let \(\norm_1, \; \norm_2\) be two norms for \(V\). They are said to be equivalent when
    \begin{equation*}
        \exists c_1,c_2 > 0 \; \forall x : \qquad c_1 \norm[x]_1 \leq \norm[x]_2 \leq c_2 \norm[x]
    \end{equation*}
\end{definition}

To check if the above definition is indeed an equivalence relation, we must show that following:
\begin{description}
    \item [Reflexive] \(\norm_1 \sim \norm_1\).
    \item [Symmetric] \(\norm_1 \sim \norm_2 \implies \norm_2 \sim \norm_1\).
    \item [Transitive] \( \norm_1 \sim \norm_2 , \; \norm_2 \sim \norm_3 \implies \norm_1 \sim \norm_3\).
\end{description}

\begin{theorem}
    All norms defined on a finite dimensional vector spacev \(V\) are equivalent.
\end{theorem}



\begin{proof}
    Let \(\norm\) be an arbitrary norm on \(V\) and \(\{e_1, e_2, \dots , e_n\} \) be a basis of \(V\). Let \(\norm_2\) be \(L_2\)-norm (Euclidean norm). It will suffice to show \(\norm \sim \norm_2\). Let
    \begin{equation*}
        M = \max \! \left( \norm[e_1], \dots , \norm[e_n] \right)
    \end{equation*}
    Take \(x \in V\), writing \(x = \sum_{i = 1}^n {\xi_i e_i}\) we have:
    \begin{equation*}
        \norm[x] = \norm[\sum_{i = i}^n {\xi_i e_i}] \leq \sum_{i = 1}^n \abs{\xi_i} \norm[e_i] \leq M \sqrt{n} \norm[x]_2
    \end{equation*}
    Taking \(c_2 = M \sqrt{n}\) proves the right inequality. For the left inequality we need the following lemma
    \begin{lemma} \label{lm:ContinuityOfNorm}
        If \(V\) is a normed vector space with \(\norm_2\), as defined above, is viewed as metric space \(\metricSpace{V}{\norm_2}\) then \(\norm : V \to \Reals\) is continuous.
    \end{lemma}

    \begin{prooflemma}
        Let \(x_0 \in V\) and \(M\) be defined as above. For any \(\epsilon > 0\) consider \(\delta = \frac{\epsilon}{M \sqrt{n}}\) then if \(\norm[x - x_0]_2 < \delta\)
        \begin{equation*}
            \abs[\norm[x] - \norm{x_0}] \leq \norm[x - x_0] \leq M \sqrt{n} \norm[x - x_0] \leq \epsilon
        \end{equation*}
    \end{prooflemma}

    Now consider the sphere of radius \(r = 1\) centered at \(0\), \(\func{S_1}{0} = S_1 = \{x \in V : \norm[x]_2 = 1\}\). One can show that \(S\) is compact. Therefore, \(\norm[x]\) assumes its minimum on \(S\). Let \( a = \norm[x_0]\) be the minimum. Since \(0 \notin S\) then \(a > 0\). By letting \(y = x / \norm[x]_2 \), we have \(y \in S\) and thus \(a \leq \norm[y]\) which is
    \begin{equation*}
        a \norm[x]_2 \leq \norm[x]
    \end{equation*}
    Taking \(c_1 = a\) proves the theorem.
\end{proof}

\begin{theorem}
    Let \(f : \Reals^n \to \Reals^n\) linear transformation is invertible if and only if there exists a \(c\) such that:
    \begin{equation*}
        c \norm[x] \leq \norm[\func{f}{x}]
    \end{equation*}
\end{theorem}

\begin{proof}
    A linear transformation \(f: \Reals^n \to \Reals^n\) is one-to-one if and only if it is surjective because \(\dim \Image f + \dim \ker f = n\). Hence, we only need to show that \(f\) is one-to-one.
\end{proof}