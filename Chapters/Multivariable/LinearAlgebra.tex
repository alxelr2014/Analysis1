\section{Linear Algebra}
\subsection{Vector Spaces}

\begin{definition}[Normed vector space]
    Let \(V\) be a vector space. A \textbf{norm} is a real valued function \(\norm: V \to \Reals\) which has the following properties
    \begin{enumerate}
        \item \(\forall x \in V, \; \norm[x] > 0\).
        \item \(\norm[x] = 0 \implies x = 0\).
        \item \(\forall x \in V \; \forall \alpha \in \Field, \; \norm[\alpha x] = \abs[\alpha] \norm[x]\).
        \item \(\forall x,y \in V \; \norm[x+y] \leq \norm[x] + \norm[y]\).
    \end{enumerate}
\end{definition}

Each normed vector space induces a metric space \(\metricSpace{V}{d}\) where \(\func{d}{x,y} = \norm[y - x]\).

\begin{theorem}
    In every normed space \(\normedSpace{V}{\norm}\) we have
    \begin{equation*}
        \abs[ {\norm[v] - \norm[w]}] \leq \norm[v - w]
    \end{equation*}
    Hence the norm is Lipschitz continuous.
\end{theorem}


\begin{definition}
    Assume \(V\) is a vector space and let \(\norm_1, \; \norm_2\) be two norms for \(V\). They are said to be equivalent when
    \begin{equation*}
        \exists c_1,c_2 > 0 \; \forall x : \qquad c_1 \norm[x]_1 \leq \norm[x]_2 \leq c_2 \norm[x]
    \end{equation*}
\end{definition}

To check if the above definition is indeed an equivalence relation, we must show that following:
\begin{description}
    \item [Reflexive] \(\norm_1 \sim \norm_1\).
    \item [Symmetric] \(\norm_1 \sim \norm_2 \implies \norm_2 \sim \norm_1\).
    \item [Transitive] \( \norm_1 \sim \norm_2 , \; \norm_2 \sim \norm_3 \implies \norm_1 \sim \norm_3\).
\end{description}

\begin{remark}
    Equivalent norms induce equivalent metrics, hence they induce the same topology.
\end{remark}

\begin{theorem} \label{th:normsEquivalent}
    All norms defined on a finite dimensional vector space \(V\) are equivalent.
\end{theorem}

\begin{proof}
    Let \(\norm\) be an arbitrary norm on \(V\) and \(\{e_1, e_2, \dots , e_n\} \) be a basis of \(V\). Let \(\norm_2\) be \(L_2\)-norm (Euclidean norm). It will suffice to show \(\norm \sim \norm_2\). Let
    \begin{equation*}
        M = \max \! \left( \norm[e_1], \dots , \norm[e_n] \right)
    \end{equation*}
    Take \(x \in V\), writing \(x = \sum_{i = 1}^n {\xi_i e_i}\) we have:
    \begin{equation*}
        \norm[x] = \norm[\sum_{i = i}^n {\xi_i e_i}] \leq \sum_{i = 1}^n \abs[\xi_i] \norm[e_i] \leq M \sqrt{n} \norm[x]_2
    \end{equation*}
    Taking \(c_2 = M \sqrt{n}\) proves the right inequality. For the left inequality we need the following lemma
    \begin{lemma} \label{lm:ContinuityOfNorm}
        If \(V\) is a normed vector space with \(\norm_2\), as defined above, is viewed as metric space \(\metricSpace{V}{\norm_2}\) then \(\norm : V \to \Reals\) is continuous.
    \end{lemma}

    \begin{prooflemma}
        Let \(x_0 \in V\) and \(M\) be defined as above. For any \(\epsilon > 0\) consider \(\delta = \frac{\epsilon}{M \sqrt{n}}\) then if \(\norm[x - x_0]_2 < \delta\)
        \begin{equation*}
            \abs[{\norm[x] - \norm[x_0]}] \leq \norm[x - x_0] \leq M \sqrt{n} \norm[x - x_0] \leq \epsilon
        \end{equation*}
    \end{prooflemma}

    Now consider the sphere of radius \(r = 1\) centered at \(0\), \(\func{S_1}{0} = S_1 = \{x \in V : \norm[x]_2 = 1\}\). One can show that \(S\) is compact. Therefore, \(\norm[x]\) assumes its minimum on \(S\). Let \( a = \norm[x_0]\) be the minimum. Since \(0 \notin S\) then \(a > 0\). By letting \(y = x / \norm[x]_2 \), we have \(y \in S\) and thus \(a \leq \norm[y]\) which is
    \begin{equation*}
        a \norm[x]_2 \leq \norm[x]
    \end{equation*}
    Taking \(c_1 = a\) proves the theorem.
\end{proof}


% \begin{theorem}
%    Let \(\normedSpace{V}{\norm}\) be a normed space over a normed complete field \(\Field\). The following are equivalent
%    \begin{enumerate}
%        \item \(V\) is finite dimensional.
%        \item every bounded closed set in \(V\) is compact.
%        \item the closed unit ball in \(V\) is compact.
%    \end{enumerate}
% \end{theorem}
% do the proof, 3 to 1 is the hardest part
% \begin{proof}

% \end{proof}

\begin{example}
    The closed unit ball in the infinite dimensional vector space \(\func{C}{\clcl{0}{1},\Reals}\) with \(\norm[f] = \max \func{f}{x}\) is not compact.  Take \(\func{f_n}{x} = x^n\). Obviously \(\norm[f_n] = 1\), however \(f_n\) doesn't uniformly converge and hence \(f_n\) doesn't have a limit in \(\func{C}{\clcl{0}{1},\Reals}\) with the \(\max\) norm. Consider the following norm
    \begin{equation*}
        \norm[f]_I = \int_0^1 \abs[\func{f}{x}] \diffOperator x
    \end{equation*}
    Note that \(\norm_I\) and \(\norm_{\max} \) are not equivalent. Let \(\func{g}{x} = 0\) for all \(x \in \clcl{0}{1}\). Then
    \begin{equation*}
        \norm[f_n - g]_I = \dfrac{1}{n+1} \to 0 \quad \text{as} n \to \infty.
    \end{equation*}
\end{example}

\begin{definition}[Banach space]
    A normed vector space \(V\) that is complete is a \textbf{Banach space}. A \textbf{Hilbert Space} is a Banach space whose norm is induced by an inner product.
\end{definition}

%TODO: proof
\begin{proposition}
    A normed finite dimensional vecotr space \(V\), is Banach space.
\end{proposition}

\begin{proof}

\end{proof}

\subsection{Linear Maps}
Let \(V\) and \(W\) be a vector spaces over \(\Field\). A map \(T: V \to W\) is \textbf{linear} if
\begin{equation*}
    \func{T}{x + \lambda y} = \func{T}{x} + \lambda \func{T}{y}
\end{equation*}
for all \(x,y \in V\) and \(\lambda \in \Field\).

\begin{definition}
    Let \(\normedSpace{V}{\norm_V}\) and \(\normedSpace{W}{\norm_W}\) be normed spaces then, a linear transformation \(T : V \to W\) is \textbf{bounded} if there exists a constant \(C > 0\) such that
    \begin{equation*}
        \norm[Tv]_W \leq C \norm[v]_V
    \end{equation*}
    for all \(v \in V\). We denote the set of all linear map from \(V \to W\) as \(\func{\CalL}{V,W}\) and the set of all bounded linear maps as \(\func{\CalB}{V,W}\). If \(T \in \func{\CalL}{V,W}\) is bijective such that \(T^{-1} \in \func{\CalL}{V,W}\), then \(T\) is called an \textbf{isomorphism} and \(V,W\) are \textbf{isomorphic}. An operator \(T \in \func{\CalL}{V,W}\) is called \textbf{isometric} if \(\norm[Tv]_W = \norm[v]_V\) for all \(v \in V\).
\end{definition}

\begin{definition}
    If \(\normedSpace{V}{\norm_V},\normedSpace{W}{\norm_W}\) are normed spaces then the \textbf{operator norm} of a linear transformation \(T : V \to W\) is
    \begin{equation*}
        \norm[T] = \sup \left\{\dfrac{\norm[Tv]_W}{\norm[v]_V} \middle| v \neq 0 \right\}
    \end{equation*}
\end{definition}

\begin{proposition}
    Let \(T : U \to V\) and \(T' : V \to W\) be two linear transformations.
    \begin{equation*}
        \norm[T' \circ T] \leq \norm[T] \norm[T']
    \end{equation*}
\end{proposition}

\begin{proof}
    for an arbitrary non-zero \(x \in U\)
    \begin{equation*}
        \norm[\func{T' \circ T}{x}]_W \leq \norm[T'] \norm[Tx]_V \leq \norm[T'] \norm[T] \norm[x]_U
    \end{equation*}
    which implies
    \begin{equation*}
        \norm[T' \circ T] \leq \norm[T] \norm[T']
    \end{equation*}
\end{proof}

\begin{theorem} \label{th:linearTransformation}
    Let \(\normedSpace{V}{\norm_V}\) and \(\normedSpace{W}{\norm_W}\) be normed spaces and \(T: V \to W\) be a linear transformation. The following are equivalent
    \begin{enumerate}
        \item \(\norm[T]\) is finite. \label{it:LinearCont1}
        \item \(T\) is bounded. \label{it:LinearCont2}
        \item \(T\) is Lipschitz continuous. \label{it:LinearCont3}
        \item \(T\) is continuous at a point. \label{it:LinearCont4}
        \item \(\sup_{\norm[v]_V = 1} \norm[Tv]_W < \infty\). \label{it:LinearCont5}
    \end{enumerate}
\end{theorem}

\begin{proof}
    %TODO: fix the references
    \cref{it:LinearCont1} \(\Rightarrow\) \cref{it:LinearCont2}: Obviously
    \begin{align*}
        \dfrac{\norm[Tv]_W}{\norm[v]_V} & \leq \norm[T]            \\
        \implies \norm[Tv]_W            & \leq \norm[T] \norm[v]_V
    \end{align*}
    note that if \(v = 0\) then \(Tv = 0\) as well and thus the last inequality holds for all \(v \in V\).

    \cref{it:LinearCont2} \(\Rightarrow\) \cref{it:LinearCont3}:
    \begin{equation*}
        \norm[Tv - Tu]_W = \norm[T(u - v)]_W \leq C \norm[u - v]_V
    \end{equation*}

    \cref{it:LinearCont3} \(\Rightarrow\) \cref{it:LinearCont4}: Trivial.

    \cref{it:LinearCont4} \(\Rightarrow\) \cref{it:LinearCont5}: Let \(T\) be continuous at \(u \in V\). Then there is  a \(\delta > 0 \) such that
    \begin{equation*}
        \norm[v-u] < \delta \implies \norm[Tv - Tu]_W = \norm[T(v-u)]_W < 1
    \end{equation*}
    Now for an arbitrary non-zero \(v\) we have
    \begin{equation*}
        \norm[\left( \dfrac{\delta v}{2\norm[v]_V} + u \right) - u]_V < \delta
    \end{equation*}
    Therefore
    \begin{align*}
         & \norm[\func{T}{\dfrac{\delta v}{2\norm[v]_V}}]_W  < 1         \\
         & \norm[\func{T}{\dfrac{v}{\norm[v]_V}}]_W  < \dfrac{2}{\delta}
    \end{align*}

    \cref{it:LinearCont5} \(\Rightarrow\) \cref{it:LinearCont1}: Let \(v \in V\) be an arbitrary vector. Then
    \begin{align*}
                 & \sup \norm[\func{T}{\dfrac{v}{\norm[v]_V}}]_W < \infty \\
        \implies & \sup \dfrac{\norm[Tv]_W}{\norm[v]_W} < \infty
    \end{align*}

\end{proof}




\begin{theorem} \label{th:finiteDimensionTransformationContinuous}
    If \(V\) is a finite dimensional normed vector space then any linear transformation \(T : V \to W\) is continuous.
\end{theorem}

\begin{proof}
    Since \(V\) is finite dimensional, according to \Cref{th:normsEquivalent}, any two norms are equivalent. Hence, take \(\norm_2\) to be Euclidean norm over a basis \(\{e_1, \dots , e_n\}\). Let \(x\) be such that \(\norm[x]_2 < \delta\) for some \(\delta > 0\). Therefore, \(\abs[\xi_i] < \delta^2\)
    \begin{equation*}
        \norm[Tx]_W = \norm[\sum \xi_i \func{T}{e_i}]_W \leq \sum \abs[\xi_i] \norm[\func{T}{e_i}]_W \leq \delta^2 K
    \end{equation*}
    where \(K = \max \norm[\func{T}{e_i}]_W \). By letting \(\delta = \sqrt{\frac{\epsilon}{K}}\) we proved continuity at \(0\) and hence the continuity by \Cref{th:linearTransformation}.
\end{proof}

\begin{corollary}
    Any finite dimensional normed vector space \(V\) over a normed complete field \(\Field\) is a Banach space.
\end{corollary}

\begin{proof}
    Let \(\{e_1, \dots , e_n\}\) be a basis for \(V\) and \(\phi : V \to \Field^n\) be the representation map for the basis. Since \(\phi\) is a linear map and a bijection then \(\phi\) is homeomorphism. Consider a Cauchy sequence \(\set{v_k} \in V\) and let \(x_k = \func{\phi}{v_k}\) then by continuity of \(\phi\) and \(\phi^{-1}\) we have
    \begin{equation*}
        \abs[x_i - x_j] = \abs[\func{\phi}{v_i} - \func{\phi}{v_j}] \leq \norm[\phi] \norm[v_i - v_j] \leq \norm[\phi] \norm[\func{\phi^{-1}}{x_i} - \func{\phi^{-1}}{x_j}] \leq \norm[\phi] \norm[\phi^{-1}] \abs[x_i - x_j]
    \end{equation*}
    hence \(\set{x_k}\) are Cauchy in \(\Field^n\) which by completeness of \(\Field\) implies that they are convergent, \(x_k \to x\). Let \(v = \func{\phi^{-1}}{x}\) then by the right side of the inequality \(v_k \to v\).
\end{proof}

\begin{remark}
    As seen in the last proof, for a bijective linear transformation \(T\)
    \begin{equation*}
        1 \leq \norm[T] \norm[T^{-1}]
    \end{equation*}
\end{remark}

\begin{theorem}
    For two normed vector spaces \(V,W\), \(\normedSpace{\func{\CalB}{V,W}}{\norm[T]}\) is a normed vector space. Moreover, it is a Banach space when \(W\) is a Banach space.
\end{theorem}


\begin{proof}
    Clearly \(\func{\CalB}{V,W}\) is a vector space. For its norm \(\norm[T]\) we have
    \begin{enumerate}
        \item \(\norm[T] \geq 0\) by definition.
        \item if \(\alpha \in \Field_W\) then
              \begin{equation*}
                  \norm[\alpha T] = \sup \left\{ \dfrac{\norm[(\alpha T)v]_W}{\norm[v]_V} \middle| v \neq 0 \right\} = \abs[\alpha] \sup \left\{ \dfrac{\norm[Tv]_W}{\norm[v]_V} \middle| v \neq 0 \right\} = \abs[\alpha] \norm[T]
              \end{equation*}
        \item for the triangle inequality
              \begin{align*}
                  \norm[T_1 + T_2] & = \sup \left\{ \dfrac{\norm[(T_1 + T_2)v]_W}{\norm[v]_V} \right\}                                                     \\
                                   & \leq \sup \left\{ \dfrac{\norm[T_1v]_W + \norm[T_2v]_W}{\norm[v]_V} \right\}                                          \\
                                   & = \sup \left\{ \dfrac{\norm[T_1v]_W}{\norm[v]_V} \right\} + \sup \left\{ \dfrac{\norm[ T_2 v]_W}{\norm[v]_V} \right\} \\
                                   & = \norm[T_1] + \norm[T_2]
              \end{align*}
    \end{enumerate}
    Suppose \(W\) is a Banach space and \(\{T_i\} \in \func{\CalB}{V,W}\) is a Cauchy sequence. Then for all \(v \in V\)
    \begin{equation*}
        \forall \epsilon \, \exists N \; \suchThat \; m,n > N \implies \norm[T_m v - T_n v]_W \leq \norm[T_m - T_n]\norm[v]_V < \epsilon
    \end{equation*}
    \(\{T_i v\}\) is a Cauchy sequence. Since \(W\) is complete then \(T_i v \to Tv\) for some function \(T\). We claim that \(T\) is a bounded linear map and is the limit of \(T_i \to T\).
    \begin{align*}
        \func{T}{v + cu} & = \lim_{i \to \infty} \func{T_i}{v + cu} = \lim_{i \to \infty} T_i v + c T_i u \\
                         & = Tv + c Tu
    \end{align*}
    Note that  \( \abs[{\norm[T_m] - \norm[T_n]}] \leq \norm[T_m - T_n]\) and hence \(\norm[T_i]\) is a Cauchy in sequence in \(\Reals\) that has a limit \(t\). There exists a \(N\) such that \(\abs[{\norm[T_n] - t}] < 1\) for all \(n \geq N\).
    \begin{equation*}
        \dfrac{\norm[Tv]_W}{\norm[v]_V} = \lim_{i \to \infty}  \dfrac{\norm[T_i v]_W}{\norm[v]_V} < t + 1
    \end{equation*}
    hence \(T\) is bounded and \(T \in \func{\CalB}{V,W}\). Finally, we show that \(T_i \to T\). For an arbitrary \(v \neq 0\) and \(\epsilon > 0\) there exist \(N\) such that
    \begin{align*}
        n \geq N \implies \norm[T_i v - Tv]_W < \epsilon \norm[v]_V
    \end{align*}
    which means that
    \begin{equation*}
        \norm[T_i - T] = \sup \dfrac{\norm[T_i v - Tv]_W}{\norm[v]_V} < \epsilon
    \end{equation*}
    Therefore \(T_i \to T\) as desired.
\end{proof}

\begin{theorem}
    Let \(\normedSpace{V}{\norm}\) be a normed space. Then any linear transformation \(T: \Reals^n \to V\) is continuous. Furthermore, if \(T\) is a bijection, it is a homeomorphism.
\end{theorem}

%TODO: a better proof
\begin{proof}
    Since \(\Reals^n\) is finite then by \Cref{th:finiteDimensionTransformationContinuous}, \(T\) is continuous. Assuming \(T\) is bijective, we must show that its inverse \(T^{-1}\) is continuous as well. Since \(T\) is a bijection then \(T\) is a linear isomorphism and \(\dim V = \dim \Reals^n = n\) hence \(T^{-1}\) is a continuous map.
\end{proof}

\begin{definition}[General linear group]
    The \textbf{general linear group} of a vector space, written \(\func{\GL}{V}\) is the set of all bijective linear transformation.
\end{definition}

\begin{proposition}
    If \(V\) is a finite (also works for infinite) vector space then \(\func{\GL}{V}\) is open in \(\func{\CalL}{V,V}\), in fact, if \(f \in \func{\GL}{V}\) then the open ball centered at \(f\) with radius \(\norm[f^{-1}]^{-1}\) remains in \(\func{\GL}{V}\). Furthermore, the inverse operator \(i : \func{\GL}{V} \to \func{\GL}{V}\), \(\func{i}{T} = T^{-1}\) is continuous.
\end{proposition}

\begin{proof}
    First assume \(f = \DSOne_V\) then we prove that any linear \(g\) that \(\norm[\DSOne_V - g] < 1\) is invertible which then implies bijectivity (true for linear maps). Let \(\norm[v] = 1\) then
    \begin{equation*}
        \abs[\norm[v] - \norm[gv]] \leq \norm[v - gv] \leq \norm[\DSOne_V - g] \norm[v] < 1
    \end{equation*}
    Therefore
    \begin{equation*}
        0 < \norm[gv] < 2
    \end{equation*}
    which means \(\ker g = \set{0}\) and since \(V\) is finite then then \(g\) is invertible. For a general \(f\), we have that
    \begin{equation*}
        \norm[1 - f^{-1} \circ g] \leq \norm[f^{-1}]\norm[f -g] < 1
    \end{equation*}
    therefore \(f^{-1} \circ g\) is invertible and as a consequence \(g = f \circ f^{-1} \circ g\) is invertible. To prove inverse operator is continuous, fix \(\epsilon > 0\) then for a \(\delta > 0\) if \(\norm[T-S] < \delta\) then
    \begin{align*}
        \norm[\DSOne_V- T^{-1} \circ S]= \norm[T^{-1} \circ T  - T^{-1} \circ S]           & \leq \norm[T^{-1}] \norm[T-S] < \delta \norm[T^{-1}] \\
        \implies  \norm[T^{-1} - S^{-1}] \leq \norm[T^{-1}\circ S - \DSOne_V]\norm[S^{-1}] & < \delta \norm[T^{-1}] \norm[S^{-1}]
    \end{align*}
    note that by letting \(\delta = \norm[T^{-1}]^{-1}/2\) then
    \begin{equation*}
        \norm[S] > -\dfrac{\norm[T^{-1}]^{-1}}{2} + \norm[T] > \dfrac{\norm[T^{-1}]^{-1}}{2}
    \end{equation*}
    also if for any invertible linear map \(R\)
    \begin{equation*}
        \norm[R] > a \implies \norm[Rx] > a\norm[x] \implies \dfrac{\norm[y]}{a} = \dfrac{\norm[R\circ \func{R^{-1}}{y}]}{a} > \norm[R^{-1}y]
    \end{equation*}
    which means that \(\norm[S^{-1}] < 2 \norm[T^{-1}]\), hence by letting
    \begin{equation*}
        \delta = \min \set{\dfrac{\epsilon \norm[T^{-1}]^2}{2} , \dfrac{\norm[T^{-1}]^{-1}}{2}}
    \end{equation*}
    we proved the continuity.
\end{proof}

\begin{theorem}
    \(T : \Reals^n \to \Reals^n\) linear transformation is invertible if and only if there exists a \(c\) such that:
    \begin{equation*}
        c \norm[x] \leq \norm[Tx]
    \end{equation*}
\end{theorem}

\begin{proof}
    If \(T\) is invertible then \(T^{-1} : \Reals^n \to \Reals^n \) is bounded and thus
    \begin{equation*}
        \norm[T^{-1}x] \leq c \norm[x]
    \end{equation*}
    and since \(T\) is bijective then there exists \(y\) such that \(x = Ty\) which implies
    \begin{equation*}
        \norm[y] \leq c\norm[Ty]
    \end{equation*}
    If there exists such \(c\) then \(\norm[Tx] > 0\) for all non-zero \(x\) and hence \(\ker T  = 0\) which implies that \(T\) is a bijection and is invertible.
\end{proof}

\begin{definition}
    Let \(V_1, V_2 ,\dots , V_n\) be  normed vector spaces. Then the function \(\phi : V_1 \times \dots \times V_n \to W\) is \textbf{\(n\)-linear} if by fixing any \(n-1\) component, \(\phi\) is linear relative to the remaining component.
\end{definition}

\begin{proposition}
    If \(V_1, V_2, \dots , V_n\) are  normed vector spaces and \(\ \phi : V_1 \times \dots \times V_n \to W \) is a \(n\)-linear then the followings are equivalent
    \begin{enumerate}
        \item \(\phi\) is continuous. \label{it:continuityOfnLinear}
        \item \(\phi\) is continuous at 0. \label{it:continuityOfnLinearataPoint}
        \item \(\phi\) is bounded, that is there exists a constant \(C > 0\) such that \label{it:boundednessOfnLinear}
              \begin{equation*}
                  \norm[\func{\phi}{v_1, \dots ,v_n}]_W \leq C \norm[v_1]_{V_1} \dots \norm[v_n]_{V_n}
              \end{equation*}
    \end{enumerate}
\end{proposition}

\begin{remark}
    As oppose to linear transformation, \(n\)-linear function's continuity does not imply uniform continuity.
\end{remark}

\begin{proof}
    \Cref{it:continuityOfnLinear} \(\implies\) \Cref{it:continuityOfnLinearataPoint}: Trivial.

    \Cref{it:continuityOfnLinearataPoint} \(\implies\) \Cref{it:boundednessOfnLinear}: For the sake of contradiction, suppose \Cref{it:boundednessOfnLinear} is false. That is, for every \(k \in \natural\) there exists a point \(v_k = (v^1_k, \dots , v^n_k)\) such that
    \begin{equation*}
        \norm[\func{\phi}{v^1_k, \dots , v^n_k}]_W > n^n \norm[v^1_k]_{V_1} \dots \norm[v^n_k]_{V_k}
    \end{equation*}
    Note that \(v^m_k\) can not be zero for any \(k\) and \(m\), otherwise \(\func{\phi}{v_k} = 0 \). Define
    \begin{equation*}
        w^m_k = \dfrac{v^m_k}{n\norm[v^m_k]_{V_k}} \to 0
    \end{equation*}
    which from the continuity at 0 implies that \(w_k = (w^1_k, \dots , w^n_k) \to 0\). However,
    \begin{equation*}
        \norm[ \func{\phi}{w_k} - \func{\phi}{0}]_W > n^n \frac{1}{n} \dots \frac{1}{n} = 1
    \end{equation*}
    which is a contradiction.

    \Cref{it:boundednessOfnLinear} \(\implies\) \Cref{it:continuityOfnLinear}. Let \(v_n \to v\) and define the points
    \begin{equation*}
        \bar{v}^m_k = (v^1 , \dots , v^m, v^{m+1}_k , \dots , v^n_k) , \qquad \bar{v}^0_k = v_k
    \end{equation*}
    and \(\bar{v}^n_k = v\). Note that \(v^m_k\) are bounded for sufficiently large \(k \geq N_1\), therefore there exists \(M\) such that \(\forall m, \; \norm[v^m_k]_{V_m} \leq M\). Also, pick \(M\) such that \(\forall m, \; \norm[v^m]_{V_m} \leq M\) as well. Then
    \begin{align*}
        \norm[\func{\phi}{v_k} - \func{\phi}{v}]_W & \leq \sum_{i = 1}^n \norm[\func{\phi}{\bar{v}^{i-1}_k  }- \func{\phi}{\bar{v}^i_k}]_W                                                              \\
                                                   & = \sum_{i = 1}^n \norm[\func{\phi}{\bar{v}^{i - 1}_k - \bar{v}^{i}_k }]_W                                                                          \\
                                                   & \leq \sum_{i = 1}^n C \norm[v^1]_{V_1} \dots \norm[v^{i-1}]_{V_{i-1}} \norm[v^i_k - v^i]_{V_i} \norm[v^{i+1}_k]_{V_{i+1}} \dots \norm[v^n_k]_{V_n} \\
                                                   & \leq CM^{n-1} \sum_{i = 1}^n \norm[v^i_k - v^i]_{V_i}
        \intertext{pick \(N_2\) such that for all \(k \geq N_2\), for each \(i, \; \norm[v^i_k - v^i]_{V_i} < \frac{\epsilon}{nCM^{n-1}}\) then}
        \norm[\func{\phi}{v_k} - \func{\phi}{v}]_W & < CM^{n-1}  \sum_{i = 1}^n \frac{\epsilon}{nCM^{n-1}} = \epsilon
    \end{align*}
\end{proof}

{\Large\textbf{Exercises}}
\begin{enumerate}
    \item Show that for a linear transformation \(T\), \(\norm[T] = \sup_{\norm[v]_V \leq 1} \norm[Tv]_W\).
\end{enumerate}
\newpage

