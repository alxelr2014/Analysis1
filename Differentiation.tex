\thispagestyle{headings}
\chapter{Differentiation}
\begin{definition}
    Let \(I\) be an interval in \(\mathbb{R}\). If \(a\) is an interior point of \(I\), then we say that \(f: I \to \mathbb{R}\) is differentiable at \(a\) when the following limit exists:
    \[\lim_{x \to a} \dfrac{f(x) - f(a)}{x - a} \]
    The limit, if exists, is denoted by \(f'(a)\).
    If \(a\) is an end point and the length of the interval is greater than zero, then the limit only exists from one direction.

    Equivalently, there exists a line \(l\), not parallel to \(y\)-axis, in form of \(l : A(x) = mx + b\), that is tangent to \(f\) at \(x = a\). In this case:
    \[\lim_{x \to a}\dfrac{f(x) - [mx + b]}{x - a } = 0 \qquad A(a) = f(a) \]
\end{definition}
In a general case, two functions \(f,g\) are tangent to each other at \(x = a\) if:
\begin{equation}
    f(a) = g(a) \qquad \lim_{x \to a}{\dfrac{f(x) - g(x)}{x - a} = 0}
\end{equation}
\begin{corollary}\leavevmode
    \begin{enumerate}
        \item \(f\) is differentiable at \(a\) if it is continuous at \(a\).
        \item \label{extrma} If \(f'(a) > 0\), there exists \(\delta > 0\) such that for \(x \in \; ]a- \delta,a[\; \cap I \implies f(x) < f(a) \) and for \(x \in \; ]a, a +  \delta[\; \cap I \implies f(x) < f(a)\). And if \(f'(a) < 0\) the inequality sign are reversed. Therefore, if \(f\) has a local extremum at \(a\), then in case \(f'(a)\) exists, \(f'(a) = 0\).
    \end{enumerate}
\end{corollary}
\begin{example}
    a function that its derivate is not continuous (with \(\sin\frac{1}{x}\)).
\end{example}
\begin{theorem}[Rolle's theorem] \label{Rolle}
    Let \(f: [a,b] \to \mathbb{R}\) be a continuous and differentiable on the interval. If \(f(a) = 0, f(b) = 0\), then there exists \(c \in \; ]a,b[\) such that:
    \[f'(c) = 0\]
\end{theorem}
\begin{proof}
    If \(f \equiv 0\) on \([a,b]\) then its derivative \(f'(x) \equiv 0\) on \([a,b]\). If \(f(x) \neq 0\) for some \(x \in ]a,b[\) then it must have a non-zero maximum or minimum at some \(c \in ]a,b[\). Since \([a,b]\) is compact then by continuity of \(f\), \(f([a,b])\) is also compact in \(\mathbb{R}\) and therefore \(f\) attains its maximum or minimum. We know that at least one of its extremities must lie in \(]a,b[\), say point \(c\), hence by \Cref{extrma} \(f'(c) = 0\).
\end{proof}
\begin{theorem}[Mean value theorem]\label{MVT}
    Let \(f: [a,b] \to \mathbb{R}\) be a continuous and differentiable on the interval,then there exists \(c \in \; ]a,b[\) such that:
    \[f'(c) = \dfrac{f(b) - f(a)}{b - a}\]
\end{theorem}
\begin{proof}
    Define \(g(x) = f(x)  - f(a) - \dfrac{f(b) - f(a)}{b-a}(x-a)\). Then it is clear that \(g(a) = g(b) = 0\) and \(g\) is continous and differentiable on the interval. Then by \Cref{Rolle} there exists \(c \in ]a,b[\) such that \(g'(c) = 0\). Equivalently:
    \begin{align*}
         & g'(c) = f'(c) -  \dfrac{f(b) - f(a)}{b-a} = 0 \\
         & \implies f'(c) =  \dfrac{f(b) - f(a)}{b-a}
    \end{align*}
    which concludes the proof.
\end{proof}
\begin{corollary}[Growth Estimate]
    If \(|f'(x)| \leq M \) in \(]a,b[\) then \(f\) satisfies the global lipschitz condition for all \(x,y \in [a,b]\) \(|f(x) - f(y) | \leq M |x-y|  \).
\end{corollary}
\begin{corollary}
    Let \(f: [a,b] \to \mathbb{R}\) is continuous and \(f'(x) < 0\) (or \(f'(x) > 0\)) for all \(x \in ]a,b[\) then \(f\) is strictly increasing (or decreasing) on \([a,b]\).
\end{corollary}
\begin{theorem}
    \(f: [a,b] \to \mathbb{R}\) is continuous and differentiable on \(]a,b[\) then for \(f'(]a,b[)\) the intermediate value theorem holds and thus it is an interval.
\end{theorem}
\begin{proof}
    Let \(x_1, x_2 \in ]a,b[\ \). WLOG assume \(f'(x_1) < f'(x_2)\), we wish to prove that for all \(y^* \in ]f'(x_1), f'(x_2)[\) there is a \(x^* \in ]x_1,x_2[ \) such that \(f(x^*) = y^*\). Put \(g(x) = f(x) - y^*x\). By differentiability of \(f\) on \([a,b]\), \(g\) is differentiable on \([a,b]\). Then, \(g'(x_1) = f'(x_1) - y^* < 0\) and \(g'(x_2) =  f'(x_2) - y^* > 0\), therefore there are \(t_1,t_2 \in ]x_1,x_2[\) such that \(g(t_1) < g(x_1)\) and \(g(t_2) < g(x_2)\). Since \(g\) is continuous on \([x_1,x_2]\) then it must attains its minimum at some \(x^* \in [x_1,x_2]\). However \(x^* \) can't be \(x_1\) or \(x_2\) and hence \(x^* \in ]x_1,x_2[\). It is then easy to see that \(f'(x^*) = y^*\).
\end{proof}
\begin{definition}[Darboux continous]
    A function \(f\) is Darboux continous if it posseses the intermediate value property.
\end{definition}
For example \(f'\) of differentiable function is Darboux continuous.
\begin{theorem}[Cauchy's mean value theorem]
    \(f,g : [a,b] \to \mathbb{R}\) are continuous then there exists a \(c \in \; ]a,b[\), such that:
    \begin{equation*}
        f'(c)(g(b) - g(a)) = g'(c)(f(b) - f(a))
    \end{equation*}
\end{theorem}
\begin{proof}
    Define \(h(x) = (f(x) - f(a))(g(b) - g(a)) -  (g(x) - g(a))(f(b) - f(a))\), then clearly \(h(a) =0, h(b) =0\) and \(h(x)\) is continous and differentiable on \([a,b]\). Hence by applying the \cref{Rolle} for some \(c \in ]a,b[\) we have:
    \begin{align*}
         & h'(c)  = 0                                            \\
         & \implies f'(c)(g(b) - g(a)) -  g'(c)(f(b) - f(a)) = 0 \\
         & \implies  f'(c)(g(b) - g(a)) =  g'(c)(f(b) - f(a))
    \end{align*}
\end{proof}

\begin{theorem}[L'Hopital's rule]
    Suppose that \(\lim_{x \to a^+}{f(x)} = 0,\lim_{x \to a^+}{g(x)} = 0 \) where \(f,g\) are differentiable on a open interval \(I = ]a,b[\) for some \(b\) such that \(g'(x) \neq 0\) in \(I\) except maybe at \(x = a\) and the limit
    \begin{equation*}
        \lim_{x \to a^+}{\dfrac{f'(x)}{g'(x)}} = L
    \end{equation*}
    exists, then:
    \begin{equation*}
        \lim_{x \to a^+}{\dfrac{f(x)}{g(x)}} = L
    \end{equation*}
\end{theorem}
\begin{proof}
    For a fixed \(\epsilon > 0\) there exists a \(\delta > 0\) such that:
    \begin{equation*}
        \abs{x - a} < \delta \implies \abs{\dfrac{f'(x)}{g'(x)} - L} < \dfrac{ \epsilon}{2}
    \end{equation*}
    then since \(f(t), g(t) \to 0\) as \(t \to a\) from right side then there must be a \(t \in ]a,x[\) such that
    \begin{equation*}
        \abs{\dfrac{f(x) - f(t)}{g(x) - g(t)} - \dfrac{f(x)}{g(x)}} < \dfrac{\epsilon}{2}
    \end{equation*}
    then simply:
    \begin{align}
        \abs{\dfrac{f(x)}{g(x)} - L} & \leq \abs{\dfrac{f(x)}{g(x)} - \dfrac{f(x) - f(t)}{g(x) - g(t)} } + \abs{\dfrac{f(x) - f(t)}{g(x) - g(t)} - L} \\
                                     & < \dfrac{\epsilon}{2} + \abs{\dfrac{f'(\theta)}{g'(\theta)} - L}                                               \\
                                     & < \epsilon
    \end{align}
    Note that \(\theta \in \; ]t,x[\) and thus \(\abs{\theta - a} < \delta \)
\end{proof}
\begin{definition}[Higher order derivatives]
    \(f\) is said to be \(r_{\text{th}}\)-differentiable at \(x\) if it is differentiable \(r\) times. The \(r_{\text{th}}\) derivative of \(f\) is denoted as \(f^{(r)}\). If \(f^{(r)}\) exists for all \(r\) and \(x\) then \(f\) is said to be infinitely differentiable or smooth.
\end{definition}
\begin{definition}[Smoothness classes]
    The set of all \(f\) is continuosly \(r_{\text{th}}\)-differentiable is called class \(\mathcal{C}^r\).
\end{definition}
\begin{definition}[Taylor polynomial]
    The \(r_{\text{th}}\)-order Taylor polynomial of an \(r_{\text{th}}\)-order differentiable function \(f\) at \(x\) is
    \begin{equation*}
        P_r(x,h) =f(x) + f'(x)h +  \dfrac{f''(x)}{2}h^2 + \dots +  \dfrac{f^{(r)}(x)}{r!} h^r = \sum_{n = 0}^{r}\dfrac{f^{(n)}(x)}{n!} h^n
    \end{equation*}
\end{definition}
\begin{theorem}[Taylor approximation theorem]
    Let \(f\) be a \(r\)-differentiable function at \(x\) then:
    \begin{enumerate}
        \item
              \begin{equation*}
                  \dfrac{f(x+h) - P_r(x,h)}{h^r} \to 0 \text{ as } h \to 0
              \end{equation*}
        \item
              and \(P_r\) is the only \(r_{\text{th}}\) degree polynomial that has such property.
        \item
              Furthermore, if \(f\) is \(r\)-differentiable on an interval \(I\) for every \(x,y \in I\), there exists \(\xi\) between \(x,y\) such that:
              \begin{equation*}
                  f(y) - P_{r-1}(x,y-x) = \dfrac{f^{(r)}(\xi )}{(r)!}(y-x)^{r}
              \end{equation*}
    \end{enumerate}
\end{theorem}
\begin{proof} \leavevmode
    \begin{enumerate}
        \item
              For the base case \(r = 1\)
              \begin{align*}
                  \lim\limits_{h \to 0}{\dfrac{f(x+h) - f(x) -f'(x)h}{h}} = f'(x) - f'(x) = 0
              \end{align*}

              and by induction we prove the case \(r = n \geq 2\)
              \begin{align*}
                   & \lim\limits_{h \to 0}{\dfrac{f(x+h) - P_n(x,h)}{h^n}} = 0                                                                      \\
                   & \iff \forall \epsilon >0, \ \exists \delta > 0\ \text{such that } \ |h| < \delta \implies |f(x+h) - P_n(x,h)| < \epsilon |h^n|
              \end{align*}

              Let \(g(h) = f(x+h) - P_n(x,h)\) then since both \(f(x+h)\) and \(P_n(x,h)\) are differentiable then we apply \Cref{MVT}
              \begin{align*}
                  g(h) - g(0) & = g(h) = h(g'(c))                                                   \\
                              & = h(f'(x+c) - \sum_{k = 1}^{n}{\dfrac{f^{(k)}(x)}{(k-1)!}c^{k-1}})  \\
                              & =  h(f'(x+c) - \sum_{k = 0}^{n-1}{\dfrac{f^{(k+1)}(x)}{k!}c^{k}})   \\
                              & =  h(f'(x+c) - \sum_{k = 0}^{n-1}{\dfrac{f^{'^{(k)}}(x)}{k!}c^{k}})
              \end{align*}
              for some \(c \in ]0,h[\). Note that \(f'\) is \((n-1)\)-differentiable at \(x\) thus by induction for any \(\epsilon > 0\) there exists a \(\delta\) such that if \(c < \delta \) then:
              \begin{equation*}
                  |f'(x+c) - \sum_{k = 0}^{n-1}{\dfrac{f^{'^{(k)}}(x)}{k!}c^{k}}| < \epsilon |c^{n-1}|
              \end{equation*}

              which means
              \begin{align*}
                  |f(x+h) - P_n(x,h)| & = |g(h)| = |h| |f'(x+c) - \sum_{k = 0}^{n-1}{\dfrac{f^{'^{(k)}}(x)}{k!}c^{k}}| \\
                                      & < |h| \epsilon |c^{n-1}|< \epsilon |h^n|
              \end{align*}

              Therefore for any \(\epsilon\) if \(h < \delta\) then \(c < \delta\) and the result holds.
        \item
              Let \(Q_r(x,h)\) be another \(r_{\text{th}}\) degree polynomial such that
              \begin{equation*}
                  \lim\limits_{h \to 0}{\dfrac{f(x+h) -Q_r(x,h)}{h^r}} = 0
              \end{equation*}

              then
              \begin{equation*}
                  \lim\limits_{h \to 0}{\dfrac{P_r(x,h) -Q_r(x,h)}{h^r}} = 0
              \end{equation*}

              however this can only happen if \(Q_r(x,h) = P_r(x,h)\).
        \item
              Again for the base case \(r = 1\)
              \begin{equation*}
                  f(y) - f(x) = f'(\xi)(y-x)
              \end{equation*}

              which is the \Cref{MVT}. for \(r = n\) we have that
              \begin{equation*}
                  g(h) = f(x+h) - P_{n-1}(x,h) + Ch^n \implies g(0) = g'(0) = \dots = g^{(n-1)} =0
              \end{equation*}
              Set \(C\) such that \(g(y-x) = 0\). Then by applying \Cref{Rolle} \((n-1)\) times

              \begin{flalign*}
                  &&g(0) = g(y-x) = 0 &\implies g'(c_1) = 0 \quad c_1\in ]0,y-x[ &&\\
                  &&g'(0) = g'(c_1) = 0& \implies g'(c_2) = 0 \quad c_2 \in ]0,c_0[ &&\\
                  && &\vdots &&\\
                  &&g^{(n-2)}(0) = g^{(n-2)}(c_{n-2}) = 0 &\implies g^{(n-1)}(c_{n-1}) = 0 \quad c_{n-1} \in ]0,c_{n-2}[&&\\
                  &&g^{(n-1)}(0) = g^{(n-1)}(c_{n-1}) = 0 &\implies g^{(n)}(\xi - x) = 0 \quad \xi - x \in ]0,c_{n-1}[ \;\subset\; ]0,y-x[&&\\
                  &&\implies  g^{(n)}(\xi - x) =  f^{(n)}(\xi) + Cn! &= f^{(n)}(\xi) - \dfrac{n!}{(y-x)^n}(f(y) - P_{n-1}(x,y-x) ) = 0 &&\\
                  &&\implies f(y) - P_{n-1}(x,y-x)  &=  \dfrac{f^{(n)}(\xi)}{n!}(y-x)^n \qquad \xi \in ]x,y[&&
              \end{flalign*}
    \end{enumerate}
    which completes the proof.
\end{proof}

\begin{theorem}[Inverse function]
    Let \(I\) be an open set and \(f : I \to \mathbb{R}\) is continuous and differentiable such that its derivate is non-zero. Thus, \(f\) is either monotonic. Furthermore, it is one to one then it has a differentiable inverse \(f^{-1}\):
    \[f^{-1}(x) = \dfrac{1}{f'(f^{-1}(x))}\]
\end{theorem}
\begin{proof}
    limit algebra
\end{proof}
