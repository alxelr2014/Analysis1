\chapter{Integration}
\begin{definition}[Partition]
    \(I = [a,b] \in \mathbb{R}\). A partition of I is a finite ordered sequence of points in \(I\).
    \begin{equation*}
        P = \{x_0, x_1, \dots, x_n | a = x_0 \leq x_1 \leq \dots \leq x_n = b\}
    \end{equation*}
    A partition pair \((P,T)\) is set
    \begin{equation*}
        (P,T) = \{x_0, t_1,x_1, t_2,x_2 \dots x_{n-1}, t_n, x_n | a = x_0 \leq t_1 \leq x_1 \leq \dots \leq t_n \leq x_n = b\}
    \end{equation*}
    Moreover, define \(\|P\| = \max{(x_i - x_{i-1})}\).
\end{definition}
The Riemann sum of a function \(f\) on the interval \([a,b]\) with respect to the pair partition \((P,T)\) is:
\begin{equation*}
    R(f,P,T)= \sum_{i = 1}^{n} {f(t_i)(x_i - x_{i-1})}
\end{equation*}
\begin{definition}[Riemann Integrability]
    If there exist a number \(S\) that for all \(\epsilon > 0\) there exist a \(\delta > 0\) such that for all partition \((P,T)\) that if \(\|P\| < \delta\) implies \(|S - R(f,P,T)| < \epsilon\), then \(f\) is Riemann integrable and \(S\) is the integral of \(f\) on \(I\) denoted by
    \begin{equation*}
        \int_{a}^{b}{f}
    \end{equation*}
    . Furthermore, if \(S\) exists then it is unique.
    Denote the set of all Riemann integrable function on an interval \(I\) as \(\mathcal{R}_I\).
\end{definition}
\begin{theorem}
    Suppose \(f \in \mathcal{R}_I\) then \(f\) is bounded.
\end{theorem}
\begin{proof}
    By Riemann integrability of \(f\), for a \(\epsilon > 0\) there is \(\delta > 0\) such that for any partition pair \((P,T)\) that \(\|P\| < \delta\) then \(|S - R(f,P,T)| < \epsilon\). Consider twp partition pair \((P,T)\) and \((P,T')\) on \(I\) with  \(\|P\| < \delta\) and \(t_i = t'_i\) for all \(i\) except \(j\). Then by triangle inequality:
    \begin{align*}
         & |R(f,P,T) - R(f,P,T')| \leq |S - R(f,P,T)| + |S - R(f,P,T')| \leq 2 \epsilon       \\
         & \implies |R(f,P,T) - R(f,P,T')| = (x_j - x_{j-1})|f(t_j) - f(t'_j)| \leq 2\epsilon
    \end{align*}
\end{proof}
\begin{corollary}  \leavevmode
    \begin{enumerate}
        \item 	If \(f,g \in \mathcal{R}_I\) and \(c \in \mathbb{R}\) then \(f + cg \in \mathcal{R}_I\) and
              \begin{equation*}
                  \int_a^b{f +cg } =  \int_a^b{f} + c\int_a^b{g}
              \end{equation*}
        \item For a constant function \(f(x) = c \) its integral is \(c(b-a)\)
        \item If \(f(x) \geq 0\) then \(\int_{a}^{b}{f} \geq 0\)
    \end{enumerate}
\end{corollary}
\begin{definition}[Commen refinement]
    Let \(P_1,P_2\) be two partitions on an interval \(I\). Their common refinement \(P^* = P_1 \lor P_2\) is defined as
    \begin{equation*}
        P^* = \{z_0 \leq z_1 \leq \dots \leq z_{m} | z_i \in P_1 \cup P_2\}
    \end{equation*}
\end{definition}
\begin{definition}[Darboux Integral]
    Suppose \(f : [a,b] \to \mathbb{R}\) is a bounded function. Define the upper Darboux and lower Darboux sums with respect to a partition \(P\) as follow
    \begin{align*}
        \text{U}(f,P) & = \sum_{i = 1}^n{M_i(x_i - x_{i-1})} & M_i = \sup\limits_{x \in [x_{i-1},x_i]}{\{f(x)\}} \\
        \text{L}(f,P) & = \sum_{i = 1}^n{m_i(x_i - x_{i-1})} & m_i = \inf\limits_{x \in [x_{i-1},x_i]}{\{f(x)\}}
    \end{align*}
    Consider \(P'\) a refinement of \(P\), then the following inequalities hold:
    \begin{equation*}
        \text{L}(f,P) \leq \text{L}(f,P') \leq 	\text{U}(f,P') \leq \text{U}(f,P)
    \end{equation*}
    Therefore as the partition gets refined the upper sum decrease and the lower sum increase. Since both of these sums are bounded then by the completeness axiom the upper and lower integral
    \begin{align*}
        \overline{\int_{a}^{b}}f  & = \inf{\{\text{U}(f,P)\}} \\
        \underline{\int_{a}^{b}}f & = \sup{\{\text{L}(f,P)\}}
    \end{align*}
    exist. In case they are equal, \(f\) is said to be Darboux integrable.
\end{definition}
\begin{theorem}
    Darboux integrability is equivalent to Riemann integrability and the value of integrals are equal.
\end{theorem}

\begin{proof}
    Firstly, assume \(f\) is bounded and Darboux integrable. Equivalently, for any \(\epsilon_1 > 0\) there exists a partition \(P\) such that

    \begin{equation*}
        \text{U}(f,P) - \text{L}(f,P) <\epsilon_1
    \end{equation*}

    Let \(\|P\| = \delta_1 \) and \(0< \delta < \delta_1\) such that if a partition \(Q\) has \(\|Q\| < \delta\) then for all partition pairing \(|I - R(f,Q,T) | < \epsilon\). Consider \(P^* = Q \lor P\). It is clear that

    \begin{equation*}
        \text{U}(f,P^*) - \text{L}(f,P^*) <\epsilon_1 \quad \text{and} \quad \|P^*\| < \delta
    \end{equation*}

    To estimate \(\text{U}(f,Q)\) and \(\text{L}(f,Q)\) consider their difference with \(\text{U}(f,P^*)\) and \(\text{L}(f,P^*)\), respectively.

    \begin{equation*}
        \text{U}(f,Q) - \text{U}(f,P^*) = \sum_{i = 1}^{n}{M^Q_i (x^Q_i - x^Q_{i-1})} - \sum_{i = 1}^{n^*}{M^*_i (x^*_i - x^*_{i-1})}
    \end{equation*}

    The sums are different only in \(x^*_i \in P\). Therefore, their difference is in the intervals that are have an endpoint in \(P\) and for each of these interval the difference is \((M_j^Q - M_i^*)(x^*_i - x^*_{i-1})\), note that \(j\) is dependent on \(i\), hence

    \begin{equation*}
        \text{U}(f,Q) - \text{U}(f,P^*) = \sum_{i = 1}^{n^P}{((M_j^Q - M_i^*)(x^*_i - x^*_{i-1})} < 2Mn^P\delta
    \end{equation*}

    where \(M\) is the bound of \(f\). Similary for the lower bounds we get:

    \begin{equation*}
        \text{L}(f,P^*) - \text{L}(f,Q) = \sum_{i = 1}^{n^P}{((m_i^*- m_j^Q )(x^*_i - x^*_{i-1})} < 2Mn^P\delta
    \end{equation*}

    As a result if set \(\delta\) such that \(\text{U}(f,Q) - \text{L}(f,Q) < \epsilon\) we will be done, since for any partition \(T\) \(R(f,Q,T),I \in [ \text{L}(f,Q),\text{U}(f,Q)]\) hence \(| I - R(f,Q,T) < \epsilon\). To do so notice

    \begin{equation*}
        \text{U}(f,Q) - \text{L}(f,Q) = \text{U}(f,Q) - \text{U}(f,P^*) + \text{U}(f,P^*) - \text{L}(f,P^*)  + \text{L}(f,P^*) -  \text{L}(f,Q) < 4Mn^P\delta + \epsilon_1
    \end{equation*}

    which will be less \(\epsilon\) if

    \begin{equation*}
        \delta = \min{(\dfrac{\epsilon}{6Mn^P}, \delta_1)} , \qquad \epsilon_1 = \dfrac{\epsilon}{3}
    \end{equation*}

    Secondly, assum \(f\) is Riemann integrable. Then for any fixed \(\epsilon > 0\) then for any two pair partition \((P,T), (P,T')\) such that \(\|P\| < \delta\) then
    \begin{equation*}
        R(f,P,T) - R(f,P,T') < \dfrac{\epsilon}{3}
    \end{equation*}

    Then choose \(T\) such that
    \begin{equation*}
        \text{U}(f,P) - R(f,P,T) < \dfrac{\epsilon}{3}
    \end{equation*}
    that is, choose \(t_i\) such that
    \begin{align*}
         & M_i - f(t_i) < \dfrac{\epsilon}{3(b-a)}                                                                                                \\
         & \implies \sum_{i=1}^{n}{(M_i - f(t_i))(x_i - x_{i-1})} <\dfrac{\epsilon}{3(b-a)} \sum_{i = 1}^{n}{x_i - x_{i-1}} < \dfrac{\epsilon}{3}
    \end{align*}
    Similarly one can choose \(T'\) so that
    \begin{equation*}
        R(f,P,T')- \text{L}(f,P)  < \dfrac{\epsilon}{3}
    \end{equation*}
    Therefore:
    \begin{equation*}
        \text{U}(f,P) -  \text{L}(f,P) = \text{U}(f,P) - R(f,P,T) + R(f,P,T) - R(f,P,T') +  R(f,P,T')- \text{L}(f,P) < \epsilon
    \end{equation*}
\end{proof}
example: f and g differ in only one point.
\begin{definition}[Zero set]
    A set \(A \subset \mathbb{R}\) is a zero set if for each \(\epsilon > 0\) there is a countable covering of \(A\) of open intervals \(]a_i, b_i[\) such that:
    \begin{equation}
        \sum_{i = 1}^{\infty}{b_i - a_i} \leq \epsilon
    \end{equation}
    If a property holds for all points except those in a zero set then one says that the property holds almost everywhere.
\end{definition}

\begin{proposition}
    The following properties hold for zero sets:
    \begin{enumerate}
        \item
              Covering of \(A\) with open intervals is equivalent to covering with closed interval.
        \item
              A finit set is a zero set.
        \item
              A countable union of zero set is a zero set.
    \end{enumerate}
\end{proposition}
\begin{definition}[Oscillation]
    Suppose \(f : I \to \mathbb{R}\) where \(I\) is an interval and \(x \in I\) then the oscillation of \(f\) at \(x\) is
    \begin{align*}
        \text{Osc}(f,x) & =\limsup\limits_{t \to x} f(t) - \liminf\limits_{t \to x} f(t) \\
                        & = \lim\limits_{h \to 0} \diam{f([x-h,x+h])}
    \end{align*}
\end{definition}
\begin{proposition}
    \(f\) is continuous at \(x\) if and only if \(\text{Osc}(f,x) = 0\).
\end{proposition}
\begin{theorem}[Riemann-Lebesgue theorem]
    The function \(f\) is Riemann integrable if and only if it is bounded and the set of its discountinuities is zero set.
\end{theorem}
\begin{proof}
    First assume \(f\) is Riemann integrable. Let \(\mathcal{D}\) be the set of all its discontinuities. Moreover, \(\mathcal{D}_n = \{x | \text{Osc}(f,x) \geq \frac{1}{n}\}\). Thus it is clear that \(\mathcal{D} = \bigcup \mathcal{D}_n \). We will show that each \(\mathcal{D}_n\) is a zero set. By Riemann integrability of \(f\), for any \(\epsilon >0\) we have a partition \(P\) such that
    \begin{equation*}
        \text{U}(f,P) - \text{L}(f,P) < \dfrac{\epsilon}{n}
    \end{equation*}
    We call \([x_{i-1},x_i] \in P\) a bad interval if there exist \(x \in [x_{i-1},x_i]\) an interior point, such that \(x \in \mathcal{D}_n\).
    \begin{align*}
         & \sum_{\text{bad}}{(M_i - m_i)(x_i - x_{i-1})} < 	\text{U}(f,P) - \text{L}(f,P) < \dfrac{\epsilon}{n}                      \\
         & \dfrac{1}{n} \sum_{\text{bad}}{(x_i - x_{i-1}) } < \sum_{\text{bad}}{(M_i - m_i)( x_i - x_{i-1}) } < \dfrac{\epsilon}{n} \\
         & \qquad \implies \sum_{\text{bad}}{(x_i - x_{i-1}) } < \epsilon
    \end{align*}
    and since the endpoints are finite then \(\mathcal{D}_n\) is a zero set and therefore, \(\mathcal{D}\) is zero set.
    Seconde assume that \(f:[a,b] \to \mathbb{R}\) is bounded and \(\mathcal{D}\) is a zero set. Choose \(n\) such that:
    \begin{equation*}
        \dfrac{1}{n} < \epsilon_1
    \end{equation*}
    for \(\epsilon_1\) that is to be determined. Since \(\mathcal{D}_n \subset \mathcal{D} \) then it is a zero set as well. In other words for any \(\epsilon_2\) there is covering of \(\mathcal{D}_n\), \(I_1, I_2, \dots\) such that
    \begin{equation*}
        \sum{\diam{I_i}} < \epsilon_2
    \end{equation*}
    For any \(x \notin \mathcal{D}_n\) we know that is an open interval \(J_x\) such that \(M_{J_x} - m_{J_x} < \dfrac{1}{n}\). Let \(I = \bigcup I_i\) and  \(J = \bigcup J_x\).
    It is clear than \(I \cup J\) is a covering of \([a,b]\). Since \([a,b]\) is compact then the open covering has a Lebesgue number\(\lambda\). Let \(P\) be a partition such that \(\|P\| < \lambda\) then an interval \([x_{i-1}, x_i]\) is bad if it is wholly within a \(I_i\) and it is good if it is not.
    \begin{align*}
        \text{U}(f,P) - \text{L}(f,P) & = \sum_{\text{good}}{(M_i - m_i)(x_i - x_{i-1})}+ \sum_{\text{bad}}{(M_i - m_i)(x_i - x_{i-1})} \\
                                      & < \dfrac{1}{n} \sum_{\text{good}}{(x_i - x_{i-1})} + 2M \sum_{\text{bad}}{(x_i - x_{i-1})}      \\
                                      & < \dfrac{b-a}{n} + 2M\epsilon_2 < (b-a)\epsilon_1 +  2M\epsilon_2 = \epsilon
    \end{align*}
    by setting \(\epsilon_1 = \dfrac{\epsilon}{2(b-a)}\) and \(\epsilon_2 = \dfrac{\epsilon}{4M}\).
\end{proof}
\begin{corollary}
    \leavevmode
    \begin{enumerate}
        \item
              Any continuous function \(f:[a,b] \to \mathbb{R}\) is integrable.
              \begin{prooflemma}
                  Since there is no point of discontinuity then it is a zero set.
              \end{prooflemma}
        \item
              Any monotonic function \(f:[a,b] \to \mathbb{R}\) is integrable.
        \item
              Product of two integrable function is integrable.
    \end{enumerate}
\end{corollary}
\begin{theorem}[Fundamental theorem of calculus]
    If \(f\) is an integrable function then its indefinite integral
    \begin{equation*}
        F(x) = \int_{a}^{x}{f(t)dt}
    \end{equation*}
    is continuous at \(x\). Furthermore, its derivative is equal to \(f(x)\) at every point \(x\) that \(f\) is continuous.
\end{theorem}
\begin{definition}
    \(F(x)\) is anti-derivate of \(f(x): [a,b] \to \mathbb{R}\) if
    \begin{equation*}
        F'(x) = f(x)
    \end{equation*}
    for all \(x \in [a,b]\).
\end{definition}
\begin{corollary}
    Every continuous function has an anti-derivative.
\end{corollary}
\begin{theorem}
    Anti-derivate of a Riemann integrable function if exists differs from its indefinite integral by a constant.
\end{theorem}
